\chapter{Preliminaries}

%An~example citation: \cite{Andel07}
\textit{What we will talk about, theory}

\section{Machine learning}
The field of machine learning encompasses a broad range of algorithms and statistical methods for data processing. In his book on machine learning, Flach provides the following general definition:

\blockquote{Machine learning is the systematic study of algorithms and systems that improve their knowledge or performance with experience.} (\cite{Flach:2012:MLA:2490546}) % page number? (3)
%
% Here a detailed definition
%

The knowledge of a system is gained through learning from data. This process is referred to as the \textit{training} phase. In this process, the algorithm adjusts its \textit{parameters} according to the nature of training data. 
The result of the training process is the prediction function, which depends on learned parameters. By applying this function on previously unseen data, denominated as the `testing' data, we obtain the output of the algorithm. The result can be then compared with the with the expected output value to determine the performance. (\cite{Bishop:2006:PRM:1162264})

The character of the data varies with different machine learning problems. 
%
% Define overfitting, generalization errotedr, how to avoid. Bias vs variance.
%

\subsection{Model ensembles}
\textit{Mention learnability?}

\textit{Other definitions...}

\section{Metalearning}


\section{Evolutionary computing}
Evolutionary computing is a heuristic method of optimization inspired by Charles Darwin's theory of \textit{natural selection}. \cite{darwin} In a population, individuals with the best characteristics are most likely to reproduce, thus passing the traits to the offspring. As the evolution is repeated over several generations, the most advantageous traits predominate. This phenomenon is also called `survival of the fittest'. % Similar to Engelbrecht... So maybe rewrite it

In an evolutionary algorithm, the goal is to find the ``best'' solution to the given problem. The term `population' refers to a set of solutions encoded as chromosomes which represent the defining features of a particular solution. The `natural selection' can be then understood as a stochastic search through the space of possible chromosome values. (\cite{Engelbrecht:2007:CII:1557464}) % tohle je skoro doslova, jak moc to musim menit? Kdyz je to v podstate definice

As can be seen in algorithm \ref{alg:EA}, a genetic algorithm should define a suitable \textit{selection} method, \textit{mutation} and/or \textit{crossover} operators and the \textit{fitness function}. The algorithm terminates when some \textit{stopping condition} is met. Some commonly used criteria, as listed by Engelbrecht, are for example a limit on the number of generations, a fitness threshold or termination after no improvement is observed. (\citep{Engelbrecht:2007:CII:1557464})

\begin{algorithm}
\DontPrintSemicolon
  \KwData{population size $n$, stopping condition $c$}
  \KwResult{evolved individuals}
  \;
  $P(0) \longleftarrow$ population of size $n$

  \While{$c$ is not met}{
      \For{individual $ind$} {
         compute fitness $f(ind)$
      }
      \For{i in \Range{$n/2$}} {
         $i_1, i_2 \longleftarrow$ select two individuals
         
         \If{$p_{cx}$} {
            perform crossover
         }
         
         \If{$p_{mut}$} {
            perform mutation
         
         }
      }
      $P(n+1) \longleftarrow$ select individuals from $P(n)$   
  }
  \;
  return $P(c)$  
\caption{Evolutionary algorithm\label{alg:EA}}
\end{algorithm}

The advantage of genetic algorithms is such that there are potentially many different solutions present in every population. With well defined selection and fitness, the algorithm performs a multi-directional search. In comparison with other directed search methods, this proves to be a more robust approach. (\cite{Michalewicz:1996:GAD:229930}, \cite{Mitchell:1997:ML:541177}) % Mitchell page 260

\section{Genetic programming}
In this section, we present a subfield of evolutionary computing --- the genetic programming --- where the population is a set of computer programs. The aim of this technique is to evolve programs which provide a good solution to the given problem. There are various approaches in means of how to represent the individuals and what kind of genetic operators to use. The fitness is computed by running the program and comparing the result with the desired output. \cite{Poli:2008:FGG:1796422}
\subsection{Tree-based genetic programming}
The individuals are most frequently represented in the form of \textit{syntax trees}. Inner nodes of the tree are functions, whereas leaves are constants and variables. Both functions and constants are selected from a set of possible nodes which is provided as input to the algorithm.

% Genetic operators? Pictures?

\subsection{Developmental genetic programming}

\section{Workflows}