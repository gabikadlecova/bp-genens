\chapter*{Introduction}
\addcontentsline{toc}{chapter}{Introduction}

Over the last few years we have witnessed an enormous success of artificial
intelligence. Thanks to advances in machine learning, a great variety of
applications emerged, ranging from automatized analyses of documents, through
image recognition to music recommendation. The artificial intelligence is also
present in medical research and has initiated the Industry 4.0. As so, it
already influences many parts of our everyday lives.

Most of these examples are however complex systems developed by large teams of
data scientists and machine learning experts. Every machine learning algorithm
depends on a set of hyperparameters which influence the learning process.
Because of that, to obtain good results they must be carefully chosen,
otherwise the algorithm fails to capture important relations in data. The
design of a machine learning model is therefore a time-consuming process
which requires lots of fine-tuning often done by trial and error. 

The character of the input data matters as well, sometimes
a series of preprocessing must be done before it is possible to analyse it.
Moreover, powerful machine learning methods, like neural networks, are usually
very computationally demanding, which further complicates the process.

As such, for small teams and non-experts it may be very difficult (or even
impossible) to develop a working machine learning system. Due to limitations
in time, budget or knowledge, instead of choosing the model best suited for
their purpose, they resort to simple methods with default settings. In some
cases, this approach may not even produce any satisfactory results.

The automated machine learning (AutoML) is a recent area of research that aims
to overcome these problems. It is a field that opens up the world of machine
learning to more people and facilitates the work of machine learning experts
as well. It encompasses systems that automatize a part of a machine learning
workflow, which is the iterative process of solving the given problem. Some
methods optimize the whole workflow, thus enabling the user to obtain decent
results even without any adjustments or data preprocessing. Other are aimed
at facilitating the model selection by proposing some models which can be
further optimized by experts.

Nevertheless, the results of AutoML systems are not always better
than hand-designed models. The reason is that in order to limit the running time
without impairing the result, compromises must be made. Existing systems
usually limit the architecture of the model or operate only with a small number
of methods. Although this approach works relatively well, it is not possible to
discover novel structures.

The goal of this work is to solve some of these limitations. We want to design
an extensive system that enables to create more complex model architectures
while also taking into account simpler structures with good results.
To do so, we explore existing systems and propose a a more general
representation. A machine learning model has the structure of a pipeline ---
a directed acyclic graph with a logical order of methods. As it is hard to
optimize this kind of structure directly, we provide an encoding that converts
an arbitrary pipeline to a tree representation.

model struct of a pipeline (meths applied one after another), 
a typic. pipe has the struc of a DAG