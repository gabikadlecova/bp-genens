\chapter{Related work} \label{ch2:related}
In this section we present existing AutoML system. The examples are divided into
two groups, where the first comprises of methods focusing mainly on simple
models and the second of systems which enable more complex architectures.
Finally we compare one of the systems (TPOT) with our approach.

\paragraph{Hyperparameter optimization methods} \label{CASH}
The first attempt to solve the CASH problem was the AutoML system \emph{Auto-WEKA}
\citep{DBLP:journals/corr/abs-1208-3719}. It employs a Bayesian optimization
method to find the optimal set of hyperparameters. To solve the model selection
at the same time, an artificial hyperparameter which represents the selected
model is added to the system. Other specific hyperparameters are induced from its
value --- for example the hyperparameters of the model itself or subestimators of
ensembles. The same principle holds also for feature preprocessing methods.

The successor of Auto-WEKA, \emph{Auto-sklearn}, improved existing systems by
adding a metalearning step to the algorithm. The optimization step is the same
as in Auto-WEKA, but the algorithm is \emph{warm-started} via a history of most
successful models on similar datasets. In the end, the resulting models are
combined together to a specific ensemble. This is done to take advantage of all
potentially good methods instead of choosing only one (not necessarily the best)
and discarding the rest. \citep{Feurer:2015:ERA:2969442.2969547}
% AutoNet --- based on autoweka, CASH for ANN, architecture as well (via
% parametrization)

Although these systems work quite well in practice, they support only a small
number of model architectures. The algorithms used in Auto-WEKA are general
enough, but the number of subestimators is limited to 5 and only base-learners can
be used. Auto-sklearn abandoned ensembles altogether in the optimization process
and does not support an arbitrary ensemble in the final step of the algorithm.
The reason behind this limitation is that by increasing the model complexity the
configuration space becomes too large.

A similar system has been developed for the purpose of automatic tuning of neural
networks --- Auto-Net. \citep{pmlr-v64-mendoza_towards_2016}

\paragraph{Architecture search}
The search for an ideal model architecture can be understood as a part of the
CASH problem. This subfield of AutoML has recently gained in popularity with the
demand for automated \emph{neural architecture search} (NAS)
\cite{2018arXiv180805377E}. It also encompasses the task of pipeline design.

There are only few systems that allow unlimited pipeline sizes. The system
\emph{RECIPE} uses grammar-based GP to evolve pipelines, but still limits the
size of the pipeline. \citep{10.1007/978-3-319-55696-3_16}
There are two systems which support arbitrary-sized pipelines. One of them is
\emph{TPOT}, which uses genetic programming to evolve tree-based models. The
root of the tree represents a single estimator whereas the branches are feature
preprocessing methods. \citep{Olson2016EvoBio}
The other system --- \emph{ML-Plan} --- is based on hierarchical planning
\citep{Mohr2018}. Both of the methods support a chain of feature preprocessors,
including feature union and stacking (ensemble methods which construct new
features from subestimators). However, none of these methods support more
complex ensemble structures.
% TODO performance estimation? TPOT newly uses subsets of features...
\paragraph{Comparison of TPOT with our system}
Our system relates most to TPOT, as it uses genetic programming to optimize
the pipelines. However, instead of using the simple GP like TPOT does, we use
the developmental GP which enables us to create more complex ensemble
structures. TPOT also focuses more on the feature preprocessing step of the 
workflow, providing additional built-in transformers beyond the scikit-learn 
provided algorithms. TPOT uses either a standard tournament selection or 
NSGA-II and multiobjective fitness. The latter is however composed from the 
score and the tree size, whereas ours uses the score and evaluation time.